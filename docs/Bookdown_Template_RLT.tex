% Options for packages loaded elsewhere
\PassOptionsToPackage{unicode}{hyperref}
\PassOptionsToPackage{hyphens}{url}
%
\documentclass[
]{book}
\usepackage{amsmath,amssymb}
\usepackage{iftex}
\ifPDFTeX
  \usepackage[T1]{fontenc}
  \usepackage[utf8]{inputenc}
  \usepackage{textcomp} % provide euro and other symbols
\else % if luatex or xetex
  \usepackage{unicode-math} % this also loads fontspec
  \defaultfontfeatures{Scale=MatchLowercase}
  \defaultfontfeatures[\rmfamily]{Ligatures=TeX,Scale=1}
\fi
\usepackage{lmodern}
\ifPDFTeX\else
  % xetex/luatex font selection
\fi
% Use upquote if available, for straight quotes in verbatim environments
\IfFileExists{upquote.sty}{\usepackage{upquote}}{}
\IfFileExists{microtype.sty}{% use microtype if available
  \usepackage[]{microtype}
  \UseMicrotypeSet[protrusion]{basicmath} % disable protrusion for tt fonts
}{}
\makeatletter
\@ifundefined{KOMAClassName}{% if non-KOMA class
  \IfFileExists{parskip.sty}{%
    \usepackage{parskip}
  }{% else
    \setlength{\parindent}{0pt}
    \setlength{\parskip}{6pt plus 2pt minus 1pt}}
}{% if KOMA class
  \KOMAoptions{parskip=half}}
\makeatother
\usepackage{xcolor}
\usepackage{color}
\usepackage{fancyvrb}
\newcommand{\VerbBar}{|}
\newcommand{\VERB}{\Verb[commandchars=\\\{\}]}
\DefineVerbatimEnvironment{Highlighting}{Verbatim}{commandchars=\\\{\}}
% Add ',fontsize=\small' for more characters per line
\usepackage{framed}
\definecolor{shadecolor}{RGB}{248,248,248}
\newenvironment{Shaded}{\begin{snugshade}}{\end{snugshade}}
\newcommand{\AlertTok}[1]{\textcolor[rgb]{0.94,0.16,0.16}{#1}}
\newcommand{\AnnotationTok}[1]{\textcolor[rgb]{0.56,0.35,0.01}{\textbf{\textit{#1}}}}
\newcommand{\AttributeTok}[1]{\textcolor[rgb]{0.13,0.29,0.53}{#1}}
\newcommand{\BaseNTok}[1]{\textcolor[rgb]{0.00,0.00,0.81}{#1}}
\newcommand{\BuiltInTok}[1]{#1}
\newcommand{\CharTok}[1]{\textcolor[rgb]{0.31,0.60,0.02}{#1}}
\newcommand{\CommentTok}[1]{\textcolor[rgb]{0.56,0.35,0.01}{\textit{#1}}}
\newcommand{\CommentVarTok}[1]{\textcolor[rgb]{0.56,0.35,0.01}{\textbf{\textit{#1}}}}
\newcommand{\ConstantTok}[1]{\textcolor[rgb]{0.56,0.35,0.01}{#1}}
\newcommand{\ControlFlowTok}[1]{\textcolor[rgb]{0.13,0.29,0.53}{\textbf{#1}}}
\newcommand{\DataTypeTok}[1]{\textcolor[rgb]{0.13,0.29,0.53}{#1}}
\newcommand{\DecValTok}[1]{\textcolor[rgb]{0.00,0.00,0.81}{#1}}
\newcommand{\DocumentationTok}[1]{\textcolor[rgb]{0.56,0.35,0.01}{\textbf{\textit{#1}}}}
\newcommand{\ErrorTok}[1]{\textcolor[rgb]{0.64,0.00,0.00}{\textbf{#1}}}
\newcommand{\ExtensionTok}[1]{#1}
\newcommand{\FloatTok}[1]{\textcolor[rgb]{0.00,0.00,0.81}{#1}}
\newcommand{\FunctionTok}[1]{\textcolor[rgb]{0.13,0.29,0.53}{\textbf{#1}}}
\newcommand{\ImportTok}[1]{#1}
\newcommand{\InformationTok}[1]{\textcolor[rgb]{0.56,0.35,0.01}{\textbf{\textit{#1}}}}
\newcommand{\KeywordTok}[1]{\textcolor[rgb]{0.13,0.29,0.53}{\textbf{#1}}}
\newcommand{\NormalTok}[1]{#1}
\newcommand{\OperatorTok}[1]{\textcolor[rgb]{0.81,0.36,0.00}{\textbf{#1}}}
\newcommand{\OtherTok}[1]{\textcolor[rgb]{0.56,0.35,0.01}{#1}}
\newcommand{\PreprocessorTok}[1]{\textcolor[rgb]{0.56,0.35,0.01}{\textit{#1}}}
\newcommand{\RegionMarkerTok}[1]{#1}
\newcommand{\SpecialCharTok}[1]{\textcolor[rgb]{0.81,0.36,0.00}{\textbf{#1}}}
\newcommand{\SpecialStringTok}[1]{\textcolor[rgb]{0.31,0.60,0.02}{#1}}
\newcommand{\StringTok}[1]{\textcolor[rgb]{0.31,0.60,0.02}{#1}}
\newcommand{\VariableTok}[1]{\textcolor[rgb]{0.00,0.00,0.00}{#1}}
\newcommand{\VerbatimStringTok}[1]{\textcolor[rgb]{0.31,0.60,0.02}{#1}}
\newcommand{\WarningTok}[1]{\textcolor[rgb]{0.56,0.35,0.01}{\textbf{\textit{#1}}}}
\usepackage{longtable,booktabs,array}
\usepackage{calc} % for calculating minipage widths
% Correct order of tables after \paragraph or \subparagraph
\usepackage{etoolbox}
\makeatletter
\patchcmd\longtable{\par}{\if@noskipsec\mbox{}\fi\par}{}{}
\makeatother
% Allow footnotes in longtable head/foot
\IfFileExists{footnotehyper.sty}{\usepackage{footnotehyper}}{\usepackage{footnote}}
\makesavenoteenv{longtable}
\usepackage{graphicx}
\makeatletter
\def\maxwidth{\ifdim\Gin@nat@width>\linewidth\linewidth\else\Gin@nat@width\fi}
\def\maxheight{\ifdim\Gin@nat@height>\textheight\textheight\else\Gin@nat@height\fi}
\makeatother
% Scale images if necessary, so that they will not overflow the page
% margins by default, and it is still possible to overwrite the defaults
% using explicit options in \includegraphics[width, height, ...]{}
\setkeys{Gin}{width=\maxwidth,height=\maxheight,keepaspectratio}
% Set default figure placement to htbp
\makeatletter
\def\fps@figure{htbp}
\makeatother
\setlength{\emergencystretch}{3em} % prevent overfull lines
\providecommand{\tightlist}{%
  \setlength{\itemsep}{0pt}\setlength{\parskip}{0pt}}
\setcounter{secnumdepth}{5}
\usepackage{booktabs}
\ifLuaTeX
  \usepackage{selnolig}  % disable illegal ligatures
\fi
\usepackage[]{natbib}
\bibliographystyle{plainnat}
\IfFileExists{bookmark.sty}{\usepackage{bookmark}}{\usepackage{hyperref}}
\IfFileExists{xurl.sty}{\usepackage{xurl}}{} % add URL line breaks if available
\urlstyle{same}
\hypersetup{
  pdftitle={Dinamica\_Poblacional},
  hidelinks,
  pdfcreator={LaTeX via pandoc}}

\title{Dinamica\_Poblacional}
\author{}
\date{\vspace{-2.5em}2023-07-11}

\usepackage{amsthm}
\newtheorem{theorem}{Theorem}[chapter]
\newtheorem{lemma}{Lemma}[chapter]
\newtheorem{corollary}{Corollary}[chapter]
\newtheorem{proposition}{Proposition}[chapter]
\newtheorem{conjecture}{Conjecture}[chapter]
\theoremstyle{definition}
\newtheorem{definition}{Definition}[chapter]
\theoremstyle{definition}
\newtheorem{example}{Example}[chapter]
\theoremstyle{definition}
\newtheorem{exercise}{Exercise}[chapter]
\theoremstyle{definition}
\newtheorem{hypothesis}{Hypothesis}[chapter]
\theoremstyle{remark}
\newtheorem*{remark}{Remark}
\newtheorem*{solution}{Solution}
\begin{document}
\maketitle

{
\setcounter{tocdepth}{1}
\tableofcontents
}
\hypertarget{sec-vii-andean-orchid-conference-cusco-peruxfa-24-26-noviembre-2023}{%
\chapter{VII Andean Orchid Conference, Cusco, Perú, 24-26 Noviembre 2023}\label{sec-vii-andean-orchid-conference-cusco-peruxfa-24-26-noviembre-2023}}

Lugar: Universidad Nacional de San Antonio Abad del Cusco, Perú Fecha
del taller: 24-26 Noviembre 2023

\hypertarget{impartido-por}{%
\section{\texorpdfstring{\emph{Impartido por}:}{Impartido por:}}\label{impartido-por}}

\emph{Dr.~Raymond L. Tremblay:}

Universidad de Puerto Rico Presidente de Analítica Fundación, Inc

e-mail:

\begin{itemize}
\item
  \href{mailto:raymond.tremblay@upr.edu}{\nolinkurl{raymond.tremblay@upr.edu}}
\item
  \href{mailto:tremblayanaliticafun@gmail.com}{\nolinkurl{tremblayanaliticafun@gmail.com}}
\end{itemize}

\emph{Dra. Nhora Helena Ospina-Calderón}:

Pontificia Universidad Javeriana Seccional Cali Profesora-investigadora

e-mail:

\begin{itemize}
\tightlist
\item
  \href{mailto:nhospina@javerianacali.edu.co}{\nolinkurl{nhospina@javerianacali.edu.co}}
\end{itemize}

\hypertarget{duraciuxf3n-del-taller}{%
\section{Duración del taller:}\label{duraciuxf3n-del-taller}}

Tres días con 26 horas totales de Taller-teórico practico con una
experiencia de recolección de datos en el campo (8 horas) y dos días de
taller--teórico practico y análisis datos (14 horas). Instrucción en
español.

\hypertarget{certificaciuxf3n-a-nombre-de-la}{%
\section{Certificación A nombre de la}\label{certificaciuxf3n-a-nombre-de-la}}

\begin{itemize}
\item
  Universidad Nacional San Antonio Abad del Cusco (UNSAAC)
\item
  ANALITICA Fundación Inc.~(Puerto Rico)
\end{itemize}

\hypertarget{costo-del-taller-estudiantes}{%
\section{Costo del taller Estudiantes:}\label{costo-del-taller-estudiantes}}

S/. xxx.00 nuevos soles (\$ 100.00 dol.) locales.

Profesionales: S/. xxx.00 nuevos soles (\$200.00 dol.) Estudiantes
internacional y profesionales

\hypertarget{cupo-25-personas}{%
\section{Cupo: 25 personas}\label{cupo-25-personas}}

\hypertarget{introducciuxf3n}{%
\section{Introducción}\label{introducciuxf3n}}

Este taller teórico-práctico se ofrece a los estudiantes interesados en
orquídeas y a las personas interesadas en la conservación de plantas.
Los métodos se enfocarán en el uso de las matrices de Lefkovitch para
construir modelos de historia de vida que sirven a su vez para evaluar
si una población de una especie es estable, está creciendo o se está
reduciendo. Para evaluar el crecimiento de las poblaciones se utiliza el
método de marcar y recapturar (monitorear) individuos en el campo, dicho
método se aplicará para hacer Matrices de Proyección de Poblaciones
(MPP). Se dará énfasis en el manejo para el almacenamiento y
sistematización de datos. Los datos recolectados en el campo serán
utilizados para construir una base de datos que almacene las evidencias
recogidas en el campo y funcione como insumo para calcular las matrices
de transiciones. Posterior a la construcción de matrices de transición
se analizará las tasas de crecimiento (crecimiento poblacional
intrínseco), estimando los errores y los intervalos de confianza con
distribución beta para cada una de las transiciones. También se llevará
a cabo análisis de elasticidad y la proyección del tamaño de la
población. Todos los análisis serán preparados y llevados a cabo en R,
Rstudio y RMarkdown (todos programas de distribución libre).

\hypertarget{objetivo-general}{%
\section{Objetivo general:}\label{objetivo-general}}

Introducir las bases teóricas y prácticas para la recolección de datos
de campo que permiten determinar la probabilidad de extinción de una
población.

\hypertarget{objetivos-especuxedficos}{%
\section{Objetivos específicos:}\label{objetivos-especuxedficos}}

• Conocer las bases teóricas y los métodos modernos de recolección de
datos en el campo para determinar la probabilidad de extinción de una
población.

\hfill\break
• Conocer los datos básicos y su correcta manipulación para determinar
la viabilidad de una población.

• Manipular correctamente los archivos electrónicos de datos de campo
para proponer la estructura de la población y su posterior depuración
para análisis sobre diferentes formatos.

• Aprender nociones básicas de Excel, R y RStudio para editar y analizar
conjuntos de datos para el uso con análisis de MPP.

• Construir un gráfico de ciclo de vida para la especie estudiada.

• Construir modelos demográficos que permitan predecir la dinámica
poblacional en el tiempo y generar insumos importantes para el manejo y
la conservación

• Estimar los intervalos de confianza de los parámetros del ciclo de
vida de la población.

• Utilizar la información y análisis demográfico para el diagnóstico y
predicciones de la viabilidad de las poblaciones.

• Modelar el tamaño poblacional de la población de estudio para
determinar su posible riesgo de extinción

\begin{verbatim}
\end{verbatim}

\hypertarget{introducciuxf3n-1}{%
\section{Introducción}\label{introducciuxf3n-1}}

\textbf{23 noviembre- Llegada a Cusco}

\begin{itemize}
\item
  8:00 pm- 9:15pm

  \begin{itemize}
  \tightlist
  \item
    Introducción al taller y Presentación de conceptos básicos
    (Capítulo 2; Tremblay)
  \item
    Diagrama del ciclo de vida (Capítulo 3; Ospina)
  \end{itemize}
\end{itemize}

\textbf{24 noviembre: Viaje de campo}

\begin{itemize}
\item
  Salida 6:00 am - Llegada 8:15 am.

  \begin{itemize}
  \tightlist
  \item
    Llegada al sitio de muestreo campo
  \end{itemize}
\item
  8:30am- 5:00pm.

  \begin{itemize}
  \item
    - Determinación de la especie a estudiar (Ospina y Tremblay)
  \item
    - Métodos de recolección de datos - Recolección de datos
    (Capítulo 4 y 22; Ospina y Tremblay)
  \end{itemize}
\end{itemize}

\textbf{25 noviembre:}

8:30 pm-10:30 pm

\begin{itemize}
\tightlist
\item
  - Teoria:

  \begin{itemize}
  \tightlist
  \item
    Proponer un ciclo de vida (Capítulo 3)
  \item
    Quien se reproduce y como calcular la fecundidad (Capitulo 6)
  \item
    Como se ama la matriz (3 x 3) y ejercicio (Capítulo 5)

    \begin{itemize}
    \tightlist
    \item
      Subir la matriz a mano

      \begin{itemize}
      \tightlist
      \item
        Multiplicación del vector (N en el tiempo t) con la
        matriz = Nt+1
      \end{itemize}
    \end{itemize}
  \end{itemize}
\end{itemize}

10:30am -- 12:00pm

\begin{itemize}
\tightlist
\item
  - Introducción a la teoria de MPP

  \begin{itemize}
  \tightlist
  \item
    crecimiento asimtotico (Capítulo 9)
  \item
    elasticidad (Capitulo 10)
  \item
    estructura estable (Capitulo Nuevo)
  \item
    valor reproductivo (Capitulo Nuevo)
  \end{itemize}
\end{itemize}

Almuerzo 12:00- 1:00pm

1:00pm - 2:30 pm

\begin{itemize}
\tightlist
\item
  - Organización de los datos en Excel/Numbers/Sheet, estructura de
  la población (Capítulo 4 y 22; Ospina y Tremblay)
\end{itemize}

2:30pm -- 5:30pm

\begin{itemize}
\item
  Introducción a R, RStudio y RMarkdown y paquetes de análisis.
  (Tremblay)
\item
  Subir los datos a RStudio, análisis preliminar de los datos usando
  \textbf{popdemo} y \textbf{raretrans} (Ospina y Tremblay)

  \begin{itemize}
  \tightlist
  \item
    Métodos para calcular contruir la matriz (Capítulo 8, Tremblay)
  \item
    Incluir la fecundidad (Capítulo 6)
  \item
    Matriz bayesiana a priori (Capítulo 8, Tremblay)
  \item
    Calcular los indices de \textbf{Oquidea cuscanensis}

    \begin{itemize}
    \tightlist
    \item
      crecimiento asintótico (Capítulo 9)
    \item
      elasticidad (Capítulo 10)
    \item
      estructura estable (Capítulo Nuevo)
    \item
      valor reproductivo (Capítulo Nuevo)
    \end{itemize}
  \end{itemize}
\end{itemize}

\textbf{26 noviembre:}

8:30 pm-10:30 pm

\begin{itemize}
\tightlist
\item
  - Descripción histórica del uso y aplicaciones de MPP (Capítulo 2,
  16: Tremblay)
\end{itemize}

10:30am -- 12:00pm

\begin{itemize}
\item
  Dinamica transitoria/ *transfer function* (Capítulo 12; Ospina)
\item
  Dinámica, análisis de viabilidad poblacional: el futuro de la
  especie (Capítulo 9: Tremblay)
\end{itemize}

Almuerzo 12:00- 1:00pm

1:00pm -- 4:30pm

\begin{itemize}
\tightlist
\item
  Estudio de casos

  \begin{itemize}
  \tightlist
  \item
    \emph{Caladenia xxx. Terrestre con latencia}
  \item
    \emph{Dracula chimaera}. Epífita y Terrestre
  \item
    \emph{Dendrophylax lindenii}. Epífita áfila
  \item
  \end{itemize}
\end{itemize}

4:30pm -- 5:30pm

\begin{itemize}
\tightlist
\item
  Presentaciones de trabajo
\end{itemize}

\hypertarget{materiales-necesarios}{%
\section{Materiales necesarios:}\label{materiales-necesarios}}

\begin{enumerate}
\def\labelenumi{\arabic{enumi}.}
\tightlist
\item
  Computadora portátil (Mac o PC) con Excel, R y Rstudio
\end{enumerate}

\begin{itemize}
\tightlist
\item
  Los participantes pueden trabajar en parejas en caso de que sea
  difícil conseguir una computadora portatil.\\
\item
  Es necesario acudir a las sesiones teóricas con los programas y
  paquetes previamente instalados, se enviará instrucciones y brindará
  oportuna asesoría.
\end{itemize}

\hypertarget{bibliografuxeda}{%
\section{Bibliografía}\label{bibliografuxeda}}

Gascoigne Samuel J. L., Simon Rolph, Daisy Sankey, Nagalakshmi
Nidadavolu, Adrian S. Stell Pičman, Christina M. Hernández, Matthew
Philpott, Aiyla Salam, Connor Bernard, Erola Fenollosa, Jessie McLean,
Shathuki Hetti Achchige Perera, Oliver G. Spacey, Maja Kajin, Anna C.
Vinton, C. Ruth Archer, Jean H. Burns, Danielle L. Buss, Hal Caswell,
Judy P. Che-Castaldo, Dylan Z. Childs, Pol Capdevila, Aldo Compagnoni,
Elizabeth Crone, Thomas H. G. Ezard, Dave Hodgson, Owen Jones, Eelke
Jongejans Jenni McDonald, Brigitte Tenhumberg, Chelsea C. Thomas, Andrew
J. Tyre, Satu Ramula, Iain Stott, Raymond L. Tremblay, Phil Wilson,
James W. Vaupel, and Roberto Salguero-Gómez.. 2023. \textbf{A standard
protocol to report discrete stage-structured demographic information.}
Submitted to Methods in Ecology and Evolution. In press.

Stott, I., Hodgson, D. J., \& Townley, S. (2012). \textbf{Beyond sensitivity:
nonlinear perturbation analysis of transient dynamics. Methods in
Ecology and Evolution}. 3(4), 673-684. doi:
10.1111/j.2041-210X.2012.00199.x

Stott, I., Hodgson, D. J., \& Townley, S. (2012b). \textbf{Popdemo: An R
package for population demography using projection matrix analysis.
Methods in Ecology and Evolution}, 3(5), 797-802.
\url{https://doi.org/10.1111/j.2041-210X.2012.00222.x}

Stott, I., Townley, S., \& Hodgson, D. J. (2011). \textbf{A framework for
studying transient dynamics of population projection matrix models}.
Ecology Letters, 14(9), 959-970. doi: 10.1111/j.1461-0248.2011.01659.x

Tremblay, R. L., \& Hutchings, M. J. (2002). \textbf{Population dynamics in
orchid conservation: a review of analytical methods based on the rare
species Lepanthes eltoroensis}. Orchid conservation. Kota Kinabalu:
Natural History Publications (Borneo), 183-204.

Tremblay, R. L., Raventos, J., \& Ackerman, J. D. (2015). \textbf{When
stable-stage equilibrium is unlikely: integrating transient population
dynamics improves asymptotic methods}. Annals of Botany, 116(3),
381-390. \url{doi:10.1093/aob/mcv031}

Tremblay, R. L., Tyre, A. J., Pérez, M. E., \& Ackerman, J. D. (2021).
\textbf{Population projections from holey matrices: Using prior information to
estimate rare transition events}. Ecological Modelling, 447, 109526.
\url{https://doi.org/10.1016/j.ecolmodel.2021.109526}.

\textbf{Que conocemos de estas especies?}

Especies reportadas de trabajo

\begin{itemize}
\item
  Cyrtochilum cimiciferum (Rchb.f.) Dalström (Tiene una gran poblacion
  )
\item
  Cyrtochilum myanthum (Lindl.) Kraenzl.1917
\item
  \emph{Epidendrum chalmersii} Hágsater \& Ric. Fernández 2013 (endémico de
  la región Cusco)
\item
  \emph{Epidendrum syringothyrsus} Rchb.f. ex Hook.f. 1875
\item
  \emph{Pleurothallis casapensis} Lindl. 1842
\item
  \emph{Habenaria} sp.
\item
  \emph{Cyclopogon} sp.
\end{itemize}

\hypertarget{quuxe9-es-el-analisis-de-dinamica-poblacional}{%
\chapter{¿Qué es el analisis de dinamica poblacional?}\label{quuxe9-es-el-analisis-de-dinamica-poblacional}}

Por: Por RLT, Nhora Ospina, Demetria y Anne \ldots..

El objetivo de la conservación biológica es asegurar que las especies pueden sobrevivir, reproducirse y dejar progenie viable para de una generación a otra. Por consecuencia se necesita que las variables intrinsicas y extrinsicas, bioticas y abioticas de cada especies estén considerado con todas sus interacciones. Naturalmente aunque el concepto es sencillo, tomar en cuanta TODAS las posibles interacciones biológicas y abióticas es imposible.\\
El primer paso a la conservación es considerar el ambiente adecuado para cada especie. Sin duda en los ultimos 50 años en muchos paises ha habido un cambio grande en el repesto y la conservaciones de bosque, pradera, desierto y todos los biomas en general. Por ejemplo el cambio de cobertura de bosque en Puerto Rico ha aumentado de circa de 2-5\% en los años 1910 a más 40\% en el 2000 \citep{pares2008agricultural}. En general en Latino América ha habido más reforestación de deforestación \citep{aide2013deforestation} en los ultimos años, aunque varia mucho entre paises y periodo de tiempo. Para la conservación el primer paso era reconocer que los habitat necesitan ser protegido.\\
Muchos de estos nuevos hábitat son bosque segundarios, fragmentados y dominado por especies introducidas. Estos habitat por consecuencia son mayormente diferentes al ambiente natural antes de los cambios antropogénicos. El resultado, en muchas ocasiones, es que la especies de interes están reducida en números de individuos o fragmentados. Considerando esos remanentes de individuos en el habitat, son suficiente para mantener una población viable? ¿Como que uno decide que una población es viable?

En general, el conceptos de conservación es que si uno proteje los habitat las especies estarán conservadas. Pero lo que no es obvio es que la presencia de muchos individuos no es suficiente para asegurar la supervivencia de una especies a largo tiempo. Un ejemplo bien conocida es la extincción del Dodo en la isla de Mauritius y la casi extinción de una especies de arbol en la familia Sapotaceae, \emph{Calvaria major}. Para que la semillas sean viable necesitan pasar por el tracto digestivo de un pajaro para remover el encocarpo persitente de la semilla que causa ``dormancy'' en las semillas \citep{temple1977plant}. Por consecuencia nunca se puede asumir que la presencia de una especies sin tomar en cuenta las interacciones bióticas y abióticas es suficiente para sugerir que no hay riesgo de extincción.

\hypertarget{quuxe9-es-el-estudio-de-la-dinamica-poblacional}{%
\section{¿Qué es el estudio de la dinamica poblacional?}\label{quuxe9-es-el-estudio-de-la-dinamica-poblacional}}

La dinámica poblacional tiene como meta tomar en cuenta todas las etapas/edades de una especies y evaluar cual de esas etapas/edades tiene impacto sobre la supervivencia de la especies. Esas etapas de vida debería considerar las interaciones con sus ambiente biotico y abiotico. La dinamica de población es fundamental en todas las areas de la ecología y evolución. Comprender la dinamica poblacional es la clave para entender la importancia relativa al aceso de los recursos y el efecto de competencia, herbivoria y depredaciones sobre la viablidad de especies. Tradicionalmente los estudios estaban enfocado a evaluar la tabla de vida para el manejo y conservaciones de especies particulares (ref). En años más recientes los estudios se han diversificado para evaluar la interacciones entre especies y su ambiente (ref).

\hypertarget{definiciuxf3n}{%
\section{Definición}\label{definiciuxf3n}}

Una definición más especifica de los estudios de dinámica poblacional son definidos como los análisis de los factores que afecten el crecimiento, estabilidad y reducción en el tamaño de la población en una serie de tiempo.

Por ejemplo, la dinamica poblacional de especies invasivas incluye un periodo de crecimiento muy lento al comienzo de la colonización de un nuevo sitio y frecuentamente siguido de un crecimiento logarithmico. La figura \ref{fig:Pop-fig}. demuestra el cambio de número de individuos en el tiempo de una especie hipotética.

\begin{Shaded}
\begin{Highlighting}[]
\FunctionTok{ggplot}\NormalTok{(pressure, }\FunctionTok{aes}\NormalTok{(temperature, pressure))}\SpecialCharTok{+}
  \FunctionTok{geom\_point}\NormalTok{()}\SpecialCharTok{+}
\NormalTok{  rlt\_theme}\SpecialCharTok{+}
  \FunctionTok{xlab}\NormalTok{(}\StringTok{"Tiempo"}\NormalTok{)}\SpecialCharTok{+}
  \FunctionTok{ylab}\NormalTok{(}\StringTok{"Tamaño poblacional"}\NormalTok{)}
\end{Highlighting}
\end{Shaded}

\begin{figure}
\centering
\includegraphics{Bookdown_Template_RLT_files/figure-latex/Pop-fig-1.pdf}
\caption{\label{fig:Pop-fig}Cambio poblacional en tiempo}
\end{figure}

\hypertarget{el-analisis-de-dinuxe1mica-poblacional-y-su-uso}{%
\section{El analisis de Dinámica Poblacional y su uso}\label{el-analisis-de-dinuxe1mica-poblacional-y-su-uso}}

Determinar el tamaño poblacional en el futuro tiene muchos usos. Se puede dividir sus usos en tres grupos grandes, entender las \textbf{1)} interacciones ecológicas, \textbf{2)} manejo y conservaciones o \textbf{3)} los procesos evolutivos. Los estudios enfocado a la conservación se engloba dentro de un acercamiento de la viablidad de poblaciones. En este libro estaremos dando una introducción a cada uno de estas vertientes, pero nuestros ejemplos son una introducción al tema y no una profundización extensa de cada uno. En la table \ref{USO} vemos algunos de los usos especificos que se ha dado con la metodología de PPM.

\hypertarget{USO}{%
\subsection{Tabla: El uso potencial de la diferentes acercamiento de PPM.}\label{USO}}

NOTA IMPORTANTE: \emph{Evaluar las referencias y añadir referencias tradicionales y recientes}

\begin{longtable}[]{@{}
  >{\raggedright\arraybackslash}p{(\columnwidth - 6\tabcolsep) * \real{0.2297}}
  >{\raggedright\arraybackslash}p{(\columnwidth - 6\tabcolsep) * \real{0.2703}}
  >{\raggedright\arraybackslash}p{(\columnwidth - 6\tabcolsep) * \real{0.2297}}
  >{\raggedleft\arraybackslash}p{(\columnwidth - 6\tabcolsep) * \real{0.2703}}@{}}
\toprule\noalign{}
\begin{minipage}[b]{\linewidth}\raggedright
Categoria de Uso
\end{minipage} & \begin{minipage}[b]{\linewidth}\raggedright
Uso especifico
\end{minipage} & \begin{minipage}[b]{\linewidth}\raggedright
Referencias
\end{minipage} & \begin{minipage}[b]{\linewidth}\raggedleft
Referencias con Oquideas
\end{minipage} \\
\midrule\noalign{}
\endhead
\bottomrule\noalign{}
\endlastfoot
Manejo & Identificar las etapas or procesos demograficos claves & \citep{crouse1987stage} & ? \\
& Determinar cuantos individuos en una población es necesario para reducir la extinción & Shaffer 1981 Armbruster \& Lande 1993 & ? \\
& Determinar cuantos individuos se necesita introducir en una sitio para establecer una población viable & Bustamante 1996 & ? \\
& Determinar cuantos individuos se puede extraer si tener un impacto negativo sobre la viabilidad de una población & Nantel et al.~1996 & ? \\
& En especies invasivas determinar cuantos y cual etapas se necesita remover para controlar la población & ? & ? \\
& Determinar cuantas pobalciones se necesita para la viabilidad de una especie al nivel local o global & Lindenmayer \& Possingham 1996 & \\
Evaluación de riesgos & Evaluar el riesgo de una población & Samson 1985 & muchos \\
& Comparando el riesgo relativo de dos o más poblaciones & Allendrof et al.~1997 & ? \\
Interacciones ecologicas & Evaluar interacciones ecológicas para entender las variables importantes para la supervivencia de una población & ? & Ospina et al., 2022 \\
Procesos y patrones evolutivos & Cual de los procesos y patrones evolutivos del ciclo de vida de especies impacta su crecimiento & ? & ? \\
\end{longtable}

\hypertarget{uso-1-identificar-las-etapas-or-procesos-demogruxe1ficos-claves}{%
\subsection{USO 1: Identificar las etapas or procesos demográficos claves}\label{uso-1-identificar-las-etapas-or-procesos-demogruxe1ficos-claves}}

Identificar y conocer cuales son las etapas de vida más suceptibles a cambios abioticos y bioticos y su impacto sobre la persistencia de una población es necesario para el manejo. El ejemplo clásico en la literatura usando PPM son los trabajos sobre la dinámica poblacional de la tortuga ``boba'' o ``cabezona'' \emph{Caretta caretta} \citep{crouse1987stage}, \citep{crowder1994predicting}. Crouse y Crowder demostrarón que aun salvando TODOS los huevos de depredación, esa estrategia de manejo antropogenico iba a tener muy poco impacto en el crecimiento de la población. Lo que econtrarón es que el impacto más grande sobre el crecimiento poblacional provendría de proteger los adultos de las redes de pesca, modificando estas para que las tortugas se pueden escapar y no ahogarse en las redes. Los trabajos de Crouse y Crowder fueron pioneros en demostrar que uno podía simular diferentes escenarios basado en la historia de vida y evaluar su impacto.

Ejemplo de orquidea AQUI

\hypertarget{uso-2-determinar-cuantos-individuos-en-una-poblaciuxf3n-es-necesario-para-reducir-la-extinciuxf3n}{%
\subsection{USO 2: Determinar cuantos individuos en una población es necesario para reducir la extinción}\label{uso-2-determinar-cuantos-individuos-en-una-poblaciuxf3n-es-necesario-para-reducir-la-extinciuxf3n}}

El efecto de tamaño poblacional sobre la biología y la probabilidad de extincción es amplia \citep{shaffer1985population}, \citep{nunney1993assessing}, \citep{harris2022abundance}. ¿Cual es la probablidad de extincción de una población considerando la cantidad de individuos en cada etapa? En general lo que se observa es que menor el tamaño poblacional, N, mayor es el riesgo de extincción. Esa probabilidad de extincción puede variar si algunas etapas del ciclo de vida tiene es muy reducido o su probabilidad de sobrevivir o crecer varia. Consideramos por ejemplo en las orquídeas donde la probabilidad de que las semillas se establece, germina y crezca a a ser un juvenil es muy pequeña. Por consecuencia una nueva población de orquidea necesita considerar la cantidad de individuos que este presente pero tambien la probabilidad de tener semillas y que estas pueden crecer a ser adultos reproducible.

\hypertarget{uso-3-determinar-cuantos-individuos-se-necesita-introducir-en-una-sitio-para-establecer-una-poblaciuxf3n-viable}{%
\subsection{USO 3: Determinar cuantos individuos se necesita introducir en una sitio para establecer una población viable}\label{uso-3-determinar-cuantos-individuos-se-necesita-introducir-en-una-sitio-para-establecer-una-poblaciuxf3n-viable}}

Naturalmente, más cantidad de individuos re-introducido en un sitio mayor sera la probabilidad que la población sea viable. Pero, como todo hay un limite de tiempo y esfuerzo disponible. Por consecuencia la pregunta debería ser orientado a determinar cual es el minimo de individuos que se deberia introductir para garantizar un x porciento de suceso en el establecimiento de una nueva población.

En los ultimos años, muchas organizaciones y cientificos han comenzado a hacer re-introducción de especies en su habitat nativo y no. (ref). Algunos programa introduce especies en areas urbanas.

\begin{itemize}
\tightlist
\item
  Caladenia
\item
  Korea
\item
  one million orchids project
\item
  ????
\end{itemize}

\hypertarget{uso-4-determinar-cuantos-individuos-se-puede-extraer-sin-tener-un-impacto-negativo-sobre-la-viabilidad-de-una-poblaciuxf3n}{%
\subsection{USO 4: Determinar cuantos individuos se puede extraer sin tener un impacto negativo sobre la viabilidad de una población}\label{uso-4-determinar-cuantos-individuos-se-puede-extraer-sin-tener-un-impacto-negativo-sobre-la-viabilidad-de-una-poblaciuxf3n}}

Hay tres razones principales para la extracción de individuos de su ambiente natural.

\begin{enumerate}
\def\labelenumi{\arabic{enumi}.}
\tightlist
\item
  Obtener individuos para la conservación \emph{Ex Situ}.
\item
  Usar un grupo de individuos para su propagación.
\item
  Extraccíon para la venta sin objetivo de conservación.
\end{enumerate}

El supuesto de colectores de orquidea de su habitat naturales, tanto para la conservación de \emph{Ex situ} y el uso para la propagación es que el impacto es minimo, y no tendrá impacto a largo plazo para la supervivencia. Regresaremos sobre este punto más tarde. La historia de fanatisismo de recolección de orquideas para la venta es bien conocida ref(). Aun que uno quisiera pensar que estas extracciones son del pasado y no occuren hoy en dia, hay todavia escrupulos que extraen los plantas sin pensar al impacto que tendrá sobre la población o especie.

Pero la pregunta se tiene que hacer. Cuantos individuos y de que etapas se puede extraer de la población sin tener impacto en el crecimoento poblacional?

\hypertarget{uso-5-en-especies-invasivas-determinar-cuantos-y-cual-etapas-se-necesita-remover-para-controlar-la-poblaciuxf3n.}{%
\subsection{USO 5: En especies invasivas determinar cuantos y cual etapas se necesita remover para controlar la población.}\label{uso-5-en-especies-invasivas-determinar-cuantos-y-cual-etapas-se-necesita-remover-para-controlar-la-poblaciuxf3n.}}

\hypertarget{uso-6-evaluar-el-riesgo-de-una-poblaciuxf3n}{%
\subsection{USO 6: Evaluar el riesgo de una población}\label{uso-6-evaluar-el-riesgo-de-una-poblaciuxf3n}}

\hypertarget{uso-7-determinar-cuantas-poblaciones-se-necesita-para-la-viabilidad-de-una-especie-al-nivel-local-o-global}{%
\subsection{USO 7: Determinar cuantas poblaciones se necesita para la viabilidad de una especie al nivel local o global}\label{uso-7-determinar-cuantas-poblaciones-se-necesita-para-la-viabilidad-de-una-especie-al-nivel-local-o-global}}

Dinamica de metapoblaciones.

\hypertarget{uso-8-comparando-el-riesgo-relativo-de-dos-o-muxe1s-poblaciones}{%
\subsection{USO 8: Comparando el riesgo relativo de dos o más poblaciones}\label{uso-8-comparando-el-riesgo-relativo-de-dos-o-muxe1s-poblaciones}}

\hypertarget{uso-9-evaluar-interacciones-ecoluxf3gicas-para-entender-las-variables-importantes-para-la-supervivencia-de-una-poblaciuxf3n}{%
\subsection{USO 9: Evaluar interacciones ecológicas para entender las variables importantes para la supervivencia de una población}\label{uso-9-evaluar-interacciones-ecoluxf3gicas-para-entender-las-variables-importantes-para-la-supervivencia-de-una-poblaciuxf3n}}

\hypertarget{uso-10-cual-de-los-procesos-y-patrones-evolutivos-del-ciclo-de-vida-de-especies-impacta-su-crecimiento}{%
\subsection{USO 10: Cual de los procesos y patrones evolutivos del ciclo de vida de especies impacta su crecimiento}\label{uso-10-cual-de-los-procesos-y-patrones-evolutivos-del-ciclo-de-vida-de-especies-impacta-su-crecimiento}}

\hypertarget{historia-de-dinamica-poblacional-en-orquideas.}{%
\section{Historia de dinamica poblacional en orquideas.}\label{historia-de-dinamica-poblacional-en-orquideas.}}

\hypertarget{referencias}{%
\section{Referencias}\label{referencias}}

\hypertarget{intro}{%
\chapter{Nombre del capitulo.}\label{intro}}

All chapters start with a first-level heading followed by your chapter title, like the line above. There should be only one first-level heading (\texttt{\#}) per .Rmd file.

\hypertarget{a-section}{%
\section{A section}\label{a-section}}

\hypertarget{otra-seccion}{%
\section{Otra seccion}\label{otra-seccion}}

All chapter sections start with a second-level (\texttt{\#\#}) or higher heading followed by your section title, like the sections above and below here. You can have as many as you want within a chapter.

\hypertarget{a-subsection}{%
\subsection*{A subsection}\label{a-subsection}}
\addcontentsline{toc}{subsection}{A subsection}

The subtopic

\hypertarget{section}{%
\subsubsection{}\label{section}}

More subdivision

\hypertarget{section-1}{%
\paragraph{}\label{section-1}}

Even more subdivision

\hypertarget{an-unnumbered-section}{%
\subsection*{An unnumbered section}\label{an-unnumbered-section}}
\addcontentsline{toc}{subsection}{An unnumbered section}

Chapters and sections are numbered by default. To un-number a heading, add a \texttt{\{.unnumbered\}} or the shorter \texttt{\{-\}} at the end of the heading, like in this section.

Remember not to use only 1 \emph{\#} as this indicates a new chapter

\hypertarget{note-the-size-of-font-changes-with-the-number}{%
\section{\texorpdfstring{NOTE the size of font changes with the number \emph{\#}}{NOTE the size of font changes with the number \#}}\label{note-the-size-of-font-changes-with-the-number}}

\hypertarget{note-the-size-of-font-changes-with-the-number-1}{%
\subsection{\texorpdfstring{NOTE the size of font changes with the number \emph{\#}}{NOTE the size of font changes with the number \#}}\label{note-the-size-of-font-changes-with-the-number-1}}

\hypertarget{note-the-size-of-font-changes-with-the-number-2}{%
\subsubsection{\texorpdfstring{NOTE the size of font changes with the number \emph{\#}}{NOTE the size of font changes with the number \#}}\label{note-the-size-of-font-changes-with-the-number-2}}

\hypertarget{note-the-size-of-font-changes-with-the-number-3}{%
\paragraph{\texorpdfstring{NOTE the size of font changes with the number \emph{\#}}{NOTE the size of font changes with the number \#}}\label{note-the-size-of-font-changes-with-the-number-3}}

Don't miss Table \ref{tab:nice-table}.

\hypertarget{cross}{%
\chapter{Cross-references}\label{cross}}

Cross-references make it easier for your readers to find and link to elements in your book.

\hypertarget{chapters-and-sub-chapters}{%
\section{Chapters and sub-chapters}\label{chapters-and-sub-chapters}}

There are two steps to cross-reference any heading:

\begin{enumerate}
\def\labelenumi{\arabic{enumi}.}
\item
  Label the heading: \texttt{\#\ Hello\ world\ \{\#nice-label\}}.

  \begin{itemize}
  \item
    Leave the label off if you like the automated heading generated based on your heading title: for example, \texttt{\#\ Hello\ world} = \texttt{\#\ Hello\ world\ \{\#hello-world\}}.
  \item
    To label an un-numbered heading, use: \texttt{\#\ Hello\ world\ \{-\#nice-label\}} or \texttt{\{\#\ Hello\ world\ .unnumbered\}}.
  \end{itemize}
\item
  Next, reference the labeled heading anywhere in the text using \texttt{\textbackslash{}@ref(nice-label)}; for example, please see Chapter \ref{intro}.

  \begin{itemize}
  \tightlist
  \item
    If you prefer text as the link instead of a numbered reference use: \protect\hyperlink{cross}{any text you want can go here}.
  \end{itemize}
\end{enumerate}

\begin{center}\rule{0.5\linewidth}{0.5pt}\end{center}

\hypertarget{captioned-figures-and-tables}{%
\section{Captioned figures and tables}\label{captioned-figures-and-tables}}

Figures and tables \emph{with captions} can also be cross-referenced from elsewhere in your book using \texttt{\textbackslash{}@ref(fig:chunk-label)} and \texttt{\textbackslash{}@ref(tab:chunk-label)}, respectively.

See Figure \ref{fig:nice-fig}.

\begin{Shaded}
\begin{Highlighting}[]
\FunctionTok{par}\NormalTok{(}\AttributeTok{mar =} \FunctionTok{c}\NormalTok{(}\DecValTok{4}\NormalTok{, }\DecValTok{4}\NormalTok{, .}\DecValTok{1}\NormalTok{, .}\DecValTok{1}\NormalTok{))}
\FunctionTok{plot}\NormalTok{(pressure, }\AttributeTok{type =} \StringTok{\textquotesingle{}b\textquotesingle{}}\NormalTok{, }\AttributeTok{pch =} \DecValTok{19}\NormalTok{)}
\end{Highlighting}
\end{Shaded}

\begin{figure}

{\centering \includegraphics[width=0.8\linewidth]{Bookdown_Template_RLT_files/figure-latex/nice-fig-1} 

}

\caption{Here is a nice figure!}\label{fig:nice-fig}
\end{figure}

Don't miss Table \ref{tab:nice-table}.

\begin{Shaded}
\begin{Highlighting}[]
\NormalTok{knitr}\SpecialCharTok{::}\FunctionTok{kable}\NormalTok{(}
  \FunctionTok{head}\NormalTok{(pressure, }\DecValTok{10}\NormalTok{), }\AttributeTok{caption =} \StringTok{\textquotesingle{}Here is a nice table!\textquotesingle{}}\NormalTok{,}
  \AttributeTok{booktabs =} \ConstantTok{TRUE}
\NormalTok{)}
\end{Highlighting}
\end{Shaded}

\begin{table}

\caption{\label{tab:nice-table}Here is a nice table!}
\centering
\begin{tabular}[t]{rr}
\toprule
temperature & pressure\\
\midrule
0 & 0.0002\\
20 & 0.0012\\
40 & 0.0060\\
60 & 0.0300\\
80 & 0.0900\\
\addlinespace
100 & 0.2700\\
120 & 0.7500\\
140 & 1.8500\\
160 & 4.2000\\
180 & 8.8000\\
\bottomrule
\end{tabular}
\end{table}

\hypertarget{appendix-list-of-epiphyitc-species}{%
\chapter*{(Appendix) List of epiphyitc species}\label{appendix-list-of-epiphyitc-species}}
\addcontentsline{toc}{chapter}{(Appendix) List of epiphyitc species}

You can add parts to organize one or more book chapters together. Parts can be inserted at the top of an .Rmd file, before the first-level chapter heading in that same file.

Add a numbered part: \texttt{\#\ (PART)\ Act\ one\ \{-\}} (followed by \texttt{\#\ A\ chapter})

Add an unnumbered part: \texttt{\#\ (PART\textbackslash{}*)\ Act\ two\ \{-\}} (followed by \texttt{\#\ A\ chapter})

Add an appendix as a special kind of un-numbered part: \texttt{\#\ (APPENDIX)\ Other\ stuff\ \{-\}} (followed by \texttt{\#\ A\ chapter}). Chapters in an appendix are prepended with letters instead of numbers.

\hypertarget{appendix-list-of-terrestrial-species}{%
\chapter*{(Appendix) List of terrestrial species}\label{appendix-list-of-terrestrial-species}}
\addcontentsline{toc}{chapter}{(Appendix) List of terrestrial species}

\hypertarget{footnotes-and-citations}{%
\chapter{Footnotes and citations}\label{footnotes-and-citations}}

\hypertarget{footnotes}{%
\section{Footnotes}\label{footnotes}}

Footnotes are put inside the square brackets after a caret \texttt{\^{}{[}{]}}. Like this one \footnote{This is a footnote.}.

Let's add a second footnote. In this case we add information on the origin of matrix algebra \footnote{The term matrix was introduced by the 19th-century English mathematician James Sylvester, but it was his friend the mathematician Arthur Cayley who developed the algebraic aspect of matrices in two papers in the 1850s.}

Mi tercer footnote es filosofico \footnote{kgjljgljhggjlhgjhgljgjhl}

\hypertarget{citations}{%
\section{Citations}\label{citations}}

Reference items in your bibliography file(s) using \texttt{@key}.

For example, we are using the \textbf{bookdown} package \citep{R-bookdown} (check out the last code chunk in index.Rmd to see how this citation key was added) in this sample book, which was built on top of R Markdown and \textbf{knitr} \citep{xie2015} (this citation was added manually in an external file book.bib).
Note that the \texttt{.bib} files need to be listed in the index.Rmd with the YAML \texttt{bibliography} key.

\hypertarget{here-is-second-citation.}{%
\subsubsection{Here is second citation.}\label{here-is-second-citation.}}

Evolutionary processes in orchids are likely to be a interaction between natural selection and genetic drift \citep{tremblay2005variation}.

\hypertarget{here-is-a-third-citation}{%
\subsubsection{Here is a third citation}\label{here-is-a-third-citation}}

un articulo de Damon excepcional \citep{damon2000review}

\hypertarget{links-to-websites}{%
\section{Links to websites}\label{links-to-websites}}

The RStudio Visual Markdown Editor can also make it easier to insert citations: \url{https://rstudio.github.io/visual-markdown-editing/\#/citations}

\url{https://www.researchgate.net/profile/Raymond-Tremblay}

\hypertarget{blocks}{%
\chapter{Blocks}\label{blocks}}

\hypertarget{equations}{%
\section{Equations}\label{equations}}

Here is an equation.

\begin{equation} 
  f\left(k\right) = \binom{n}{k} p^k\left(1-p\right)^{n-k}
  \label{eq:binom} 
\end{equation}

You may refer to using \texttt{\textbackslash{}@ref(eq:binom)}, like see Equation \eqref{eq:binom}.

-- this is the script to make the equation connectable in the text

** that the \texttt{....} are to make the text visual

\hypertarget{theorems-and-proofs}{%
\section{Theorems and proofs}\label{theorems-and-proofs}}

Labeled theorems can be referenced in text using \texttt{\textbackslash{}@ref(thm:tri)}, for example, check out this smart theorem \ref{thm:tri}.

::: \{.theorem \#tri\}
For a right triangle, if \(c\) denotes the \emph{length} of the hypotenuse
and \(a\) and \(b\) denote the lengths of the \textbf{other} two sides, we have
\[a^2 + b^2 = c^2\]

A site to help create your equations \[\bar{x}=\frac{\sum x_{i}}{n}\]

\url{https://latex.codecogs.com/eqneditor/editor.php}

Ahora se enseña la formula del promedio \ref{thm:promedio}

\begin{theorem}
\protect\hypertarget{thm:promedio}{}\label{thm:promedio}\[\bar{x}= \frac{\sum x_{i}}{n}\]

Si quiere la ecuación en la linea usa solamente un ``\$'' antes y despues de la formula.
El promedio tiene la siguiente formula \(\bar{x}= \frac{\sum x_{i}}{n}\) y la varianza se estima tomando la diferencia entre los valores y el promedio.
\end{theorem}

Read more here \url{https://bookdown.org/yihui/bookdown/markdown-extensions-by-bookdown.html}.

\begin{center}\rule{0.5\linewidth}{0.5pt}\end{center}

\hypertarget{callout-blocks}{%
\section{Callout blocks}\label{callout-blocks}}

The R Markdown Cookbook provides more help on how to use custom blocks to design your own callouts: \url{https://bookdown.org/yihui/rmarkdown-cookbook/custom-blocks.html}

\hypertarget{sharing-your-book}{%
\chapter{Sharing your book}\label{sharing-your-book}}

\hypertarget{publishing}{%
\section{Publishing}\label{publishing}}

HTML books can be published online, see: \url{https://bookdown.org/yihui/bookdown/publishing.html}

\hypertarget{pages}{%
\section{404 pages}\label{pages}}

By default, users will be directed to a 404 page if they try to access a webpage that cannot be found. If you'd like to customize your 404 page instead of using the default, you may add either a \texttt{\_404.Rmd} or \texttt{\_404.md} file to your project root and use code and/or Markdown syntax.

\hypertarget{metadata-for-sharing}{%
\section{Metadata for sharing}\label{metadata-for-sharing}}

Bookdown HTML books will provide HTML metadata for social sharing on platforms like Twitter, Facebook, and LinkedIn, using information you provide in the \texttt{index.Rmd} YAML. To setup, set the \texttt{url} for your book and the path to your \texttt{cover-image} file. Your book's \texttt{title} and \texttt{description} are also used.

This \texttt{gitbook} uses the same social sharing data across all chapters in your book- all links shared will look the same.

Specify your book's source repository on GitHub using the \texttt{edit} key under the configuration options in the \texttt{\_output.yml} file, which allows users to suggest an edit by linking to a chapter's source file.

Read more about the features of this output format here:

\url{https://pkgs.rstudio.com/bookdown/reference/gitbook.html}

Or use:

\begin{Shaded}
\begin{Highlighting}[]
\NormalTok{?bookdown}\SpecialCharTok{::}\NormalTok{gitbook}
\end{Highlighting}
\end{Shaded}


  \bibliography{book.bib,packages.bib}

\end{document}
